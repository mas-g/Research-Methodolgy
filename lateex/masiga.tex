\documentclass[a4paper,12pt]{article}
\begin{document}


\begin{Huge}
\begin{center}
\begin{normalsize}

\textbf{MAKERERE UNIVERSITY } \\
\textbf{FACULTY OF COMPUTING AND INFORMATICS TECHNOLOGY} \\
\textbf{DEPARTMENT OF COMPUTER SCIENCE} \\
\textbf{BACHELOR OF SCIENCE IN COMPUTER SCIENCE} \\
\textbf{BIT 2207 RESEARCH METHODOLOGY} \\
\textbf{YEAR 2} \\


\textbf{\sc MASIGA DAVID KELVIN } \\
\textbf{\sc Reg No: 16/U/579 } \\
\textbf{\sc std No: 216000507}\\
\end{normalsize}
\end{center}
\end{Huge}
\newpage

\section{\sc YOUTH UNEMPLOYEMENT IN UGANDA.}
\paragraph{\sl  .}

\subsection{\sc Abstract}
\paragraph{ \sl Uganda has one of the youngest and fastest growing populations in the world. Youth are often described as a group of people with high potential for increasing productivity and hence can be a good basis for economic growth. However, a large young and fast-growing population also poses immense challenges in the form of widespread youth unemployment. Statistics shows that youth faces a higher unemployment rates than adults, and that women faces higher rates than men. Hence, young women face a double burden by being both youth and female. Young women often find themselves trapped in the middle between the expectations culture and society has of them as women, and their own aspirations. This thesis uses gender analysis to identify some of the challenges faced by youths, especially young women, when looking for employment. And secondly, examines the survival strategies employed by youths.}
\paragraph{\sl  .}
\subsection{\sc introduction}
\paragraph{\sl Unemployment refers to a situation where people in a working age group are available for paid employment or self-employment but there is no available work for them to do. The range of working age in Uganda is between 15-64years. Uganda’s national unemployment rate stood at 11.7 percent in October 2012. The rural unemployment rate was 1.7%. Urban unemployment rate for the youth stands at 12 percent and unemployment rate for youth was 32.2% in Kampala in 2010. Youth refers to people aged between 12 and 30 years.}
\paragraph{\sl Youth unemployment remains a serious policy challenge in many sub-Saharan African countries, including Uganda. In 2013, youth (aged 15 to 24) in sub-Saharan Africa were twice likely to be unemployed compared to any other age cohort. For Uganda, in 2012, the Uganda Bureau of Statistics revealed that the share of unemployed youth (national definition, 18-30 years) among the total unemployed persons in the country was 64 percent. Given the rapid growth of the Ugandan population—three-quarters of the population are below the age of 30 years—coupled with the fact that the youth are getting better educated through higher access to primary and secondary education, a stronger focus on job creation for this cohort of people cannot be overemphasized.}

\section{\methods}
\paragraph{\sl In this research,an applied research form of research has been conducted.}
\section{\observation}
\subsection{\sc Causes of unemployment in Uganda}
\paragraph{\sl 
The unemployment in Uganda is mainly due to the following causes:
•	Theoretical methods of teaching are used by a number of learning institutions.
•	High levels of physical disability which has rendered it impossible for thousands to perform work and there are no adequate support facilities to enable the disabled perform work.
•	The graduates inability to create jobs is another source of unemployment
•	Influx of foreign workers brought by investors.
•	Lack of support for young entrepreneurs especially in the rural areas.
•	Lack of access to resources like land and capital for a number of youth.
•	Gender or other discrimination in areas of employment for example there are ladies who are taxi drivers or conductors.
•	lack of employable skills,
•	Lack of focus by the existing programs on the informal sector and agriculture.
•	Lack of apprenticeship schemes.
•	Negative attitudes by the youth towards work especially in agriculture and rural areas.
•	Lack of a comprehensive employment policy.
•	Rapid changes in technology
•	Inflation which makes expensive to pay a living salary
.}
\subsection{\sc references}

\textbf{\sl 	

}



\end{document}